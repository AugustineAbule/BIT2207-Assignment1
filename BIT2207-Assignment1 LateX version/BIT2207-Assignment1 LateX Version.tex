\documentclass{article}
\usepackage[utf8]{inputenc}
\usepackage[T1]{fontenc}
\usepackage{geometry}
\geometry{a4paper}
\usepackage{helvet}
\renewcommand{\familydefault}{\sfdefault}
\title{A research report on Military Reconnaissance}
\author{By Abule Augustine Arumadri 215014379 15/U/2633/PS}
\date{}
\setlength{\topmargin}{-1cm}
\begin{document}
\maketitle
\tableofcontents
\section{The art of combatant spying}
Reconnaissance, commonly referred to as ‘Recon’ by military personnel, is the act of military operatives covertly deploying into hostile territory to monitor and collect 
data about enemy soldiers, study their military activities, general setup of the enemy territory and any other valuable information that can be helpful to their superiors
 in planning an effective imminent attack on the enemy with an element of surprise, which is one of the best advantages an attack squadron can have on the enemy.

 Recon missions can be done by actual soldiers on the ground, or by specialized aircraft like the United States Air Force Lockheed SR-71 Blackbird, or
 even by the modern technology of Unmanned Aerial Vehicles (UAVs) commonly referred to as drones.


\section{Eye in the sky}
In the event of use of aircraft, highly trained pilots flying specialized aircraft fitted with advanced cameras and sound equipment covertly enter enemy airspace 
and carry out surveillance to capture information in accordance with their mission objectives, but do not necessarily carry out offensive attacks but rather defensive
 attacks in case of discovery and engagement by enemy Surface-to-Air missiles (SAMs).

 Recon mission flights are usually flown at night to in order to minimize being spotted
 by the enemy, and even still, new technology has introduced stealth planes that give off a very minimal radar signature, like the Lockheed Martin F-22 Raptor, hence invisible 
to enemy radar systems. Night vision equipment are used to enhance visibility at night. Better still, UAVs are very effective in reconnaissance since they can be operated
 during day and provide the comfort of remote control and so pilots’ lives are not put at risk if the plane is shot down or experiences fatal mechanical mishaps. 

UAVs are a niche technology, fitted with a multiple spectral targeting system that is able to relay crystal clear imagery from about 30,000 feet above target to mission 
commanders 7000 miles away via satellite. Over 40 militaries worldwide employ the use of UAVs in their operations but they come in different variations and capabilities
 like the RQ-11 Raven, RQ-4 Global Hawk, X47-V, and MQ-9 Reaper which is autonomous and fully armed. UAVs are largely expensive for most countries, and so they resort 
to actual insertion of elite soldiers on ground, as it was done before the technological advancements of today.

\section{Boots on the ground}
I focused my research on the SAS (The British Special Forces), a highly secretive military organization that the identities of those who serve in it cannot be revealed.
 To join this revered Special Forces regiment, a soldier must first pass the selection course. It is arguably the most physically challenging and psychologically demanding
 military selection process in the world. “Soldiers are trained to have the courage that would make normal men crumble in fear and perseverance that will ensure they never 
accept mission failure”, says Colonel James Ryan (real name withheld), an SAS commander. “Most who start the course will never finish” he adds. Chosen from the best
 of the British military services, the candidates begin a grueling three week test, and a shot at becoming one of the world’s elite soldiers. From this moment on, their moves are scrutinized
 by the directing staff drawn from the classified ranks of the SAS, most of whom must remain disguised.  Interviews conducted by veteran and serving SAS soldiers probe for signs of
 mental weakness among the candidates. Brutal physical training sessions provide the staff with a clear picture of the capability of the men to perform their duties on actual war terrain 
when it is really cunning. “The candidates are basically on a daily renewable contract” says the venerated colonel. 

On mission day, specialized military aircraft are used to insert soldiers depending on the terrain of the landing zone (LZ), proximity to known presence of enemy soldiers and
 mode of insertion. In cases of ground deployment, modified Boeing CH-47 Chinooks or stealth Sikorsky HH-60 Pave Hawk helicopters are used. In cases of aerial deployment, 
a Boeing V-22 Osprey or C-17 Globemaster III can be used, whereby soldiers are dropped at a certain altitude over the enemy territory and then glide to the planned LZ using parachutes.
 On safe insertion, the patrol team covertly proceeds to specified locations to carry out their mission objectives and collect all the relevant information. They remain behind enemy lines for 
the specified mission duration, and they survive on the military-grade food supplies they carry with them, usually canned foods, which they like to call ready-to-eat meals.

 A typical soldier on a  recon mission carries with him fire-retardant gloves, a body armored Kevlar vest, night-vision eyewear, thermal weapon sight, lower extremity body armor, anti-ballistic eyewear among many other equipment.
 Even with all these top-grade logistics, mission success largely depends on the vital combat survival techniques that the soldiers choreograph countless times so they can appropriately and effectively react to any possible
 complication that may arise during the mission. Team work is a very big bonding factor between soldiers to trust each other with their lives.

\section{Mission gone south} 
In the event that the mission has been compromised, be it discovery by enemy soldiers or a squad member being critically injured before completion of the mission,
 a distress signal is relayed on specified radio frequencies using specific call-signs and an immediate extraction can be organized for the team.

 But in a situation experienced by now retired SAS soldier Chris Ryan in January 1991 during the Gulf War, whose eight man patrol team was tasked on a recon mission into Iraq,
 inserted 140 miles into the western Iraq desert to locate Saddam Hussein’s mobile scud missile launchers, was discovered by enemy soldiers and radio contact to remote mission commanders was lost,
 immediate extraction could not be achieved. The team had to fight its way through a gun battle against the Iraqi army insurgents and move over 200 miles to the Syrian border on foot. Only one soldier made it across the border into Syria alive. 

Such unfortunate twists of events require the maximum mental strength and effective application of world-class training techniques they underwent, and it is no surprise that countries such as the United States, the United Kingdom, Russia, China, Israel and North Korea have one of the most advanced and highly trained military outfits in the world given the massive budget allocations to their armed forces, and they will continue to compete for global military superiority. In that bid, US President Donald Trump on 28th February 2017 in his address to Congress urged them increase military funding in the next budget, to the already enormous estimated 600 billion dollar annual military expenditure, by far the largest in the world. China too is set to ramp up its military expenditure in the next budget.

\section{Conclusion} 
In conclusion, it is my opinion that military reconnaissance is a pivotal operation in any effort to achieve success in an armed confrontation between hostile nations,
 be it a coalition of countries or individual states, because it provides adversarial armies with vital information about their enemies, information which is  used in battle planning
 to achieve the ultimate goal of any military operation with an offensive objective, crushing the enemy! 

\end{document}
